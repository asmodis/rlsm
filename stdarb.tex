\documentclass[a4paper, halfparskip*]{scrartcl}

\usepackage[utf8]{inputenc}
\usepackage[T1]{fontenc}
\usepackage[german]{babel}

\usepackage{amsmath,paralist}

\begin{document}
\section{Theoretische Grundlagen}
\label{sec:tg}
In diesem Abschnitt geht es uns darum, die wichtigsten theoretischen Grundlagen
darzulegen.

\subsection{Magmen, Halbgruppen und Monoide}
\label{sec:tg:monoide}
\textbf{Defintion:} Sei $\emptyset \neq M$ eine Menge. Eine Abbildung
\begin{equation*}
  * \colon M \times M \to M,\, (a,b) \mapsto a*b
\end{equation*}
heißt \emph{(binäre) Operation auf $M$}. Eine Operation heißt \emph{assoziativ}, falls
\begin{equation*}
  \forall a,b,c\in M \colon a*(b*c) = (a*b)*c
\end{equation*}
und sie heißt \emph{kommutativ}, falls
\begin{equation*}
  \forall a,b\in M \colon a*b = b*a
\end{equation*}

Ein \emph{Magma} $M$ ist eine nichtleere Menge zusammen mit einer Operation $*$
auf $M$. Ist die Operation assoziativ, so nennen wir $M$ eine \emph{Halbgruppe}.

\textit{Bemerkung:} Sofern keine Verwirrung entstehen kann, werden wir die
Operation nicht explizit mitangeben und schreiben statt $a*b$ einfach
$ab$. Zusätzlich definieren wir für natürliche Zahlen $n\geq 1$ und ein Element
$a\in M$ rekursiv
\begin{equation*}
  a^1 := a;\quad a^n := a*a^{n-1}
\end{equation*}


\textbf{Defintion:} Sei $M$ ein Magma und $x\in M$. Dann heißt $x$
\begin{enumerate}[(a)]
\item \emph{Neutralelement}, falls $\forall a\in M\colon xa = ax = a$.
\item \emph{Links-Nullelement}, falls $\forall a\in M\colon xa = x$.
\item \emph{Rechts-Nullelement}, falls $\forall a\in M\colon ax = x$.
\item \emph{Nullelement}, falls es ein Links- und Rechts-Nullelement ist.
\item \emph{idempotent}, falls $x^2 = x$. Sind alle Elemente eines Magmas
  idempotent, so nennen wir das Magma selbst \emph{idempotent}.
\item \emph{Inverses zu $a$}, falls $ax = xa = 1$, wobei $1$ ein Neutralelement
  bezeichnet.
\end{enumerate}
Ein \emph{Monoid} ist eine Halbgruppe mit einem Neutralelement. Ein Monoid in
dem jedes Element ein Inverses besitzt heißt \emph{Gruppe}.

\textit{Bemerkung:} Falls ein Inverses, Neutral- oder Nullelement existiert,
so ist es eindeutig. Der Beweis dafür ist leicht und in jedem Lehrbuch zur
Algebra zu finden. Wir werden in Zukunft immer das Symbol $1$ für das
Neutralelement eines Monoids $M$ verwenden (oder auch $1_M$ um deutlich zu
machen, dass dies das Neutralelement von $M$ ist) und das Symbol $0$ bezeichnet
immer das Nullelement, falls es existiert. Das Inverse zu $a$ bezeichnen wir mit
$a^{-1}$ (falls es existiert).

Im Fall eines Monoids $M$ setzen wir noch $a^0 = 1$ für jedes $a\in M$.

\textbf{Definition:} Sei $M$ ein Monoid. Eine Menge $N$ heißt
\emph{Untermonoid} von $M$, in Zeichen $N\leq M$, falls
\begin{enumerate}[(i)]
\item $N \subseteq M$
\item $1_N = 1_M$
\item $\forall a,b\in N \colon a*b\in N$
\end{enumerate}

\textit{Bemerkung:} $N$ wird mit der Operation  $\circ = *|_{N\times N}$ selbst
zu einem Monoid.

\textbf{Proposition:} Der Schnitt von Untermonoiden ist wieder ein Untermonoid.

\textbf{Definition:} Sei $M$ ein Monoid und $N \subset M$. Der von $N$
\emph{erzeugte Untermonoid} ist
\begin{equation*}
  \langle N \rangle := \bigcap_{N \subset S \leq M} S
\end{equation*}

Eine Menge $E$ heißt \emph{Erzeuger} von $M$, falls $\langle E \rangle = M$.

\textbf{Defintion:} Seien $(M,*)$ und $(N,\circ)$ Monoide. Eine Abbildung $\phi\colon M \to
N$ heißt \emph{(Monoid-)Homomorphismus}, falls
\begin{enumerate}[(i)]
\item $\phi(1_M) = 1_N$
\item $\forall a,b \in M \colon \phi(a*b) = \phi(a)\circ\phi(b)$
\end{enumerate}
Ein Homomorphismus $\phi$ heißt
\begin{enumerate}[(a)]
\item \emph{Monomorphismus}, falls $\phi$ injektiv ist.
\item \emph{Epimorphismus}, falls $\phi$ surjektiv ist.
\item \emph{Isomorphismus}, falls $\phi$ bijektiv ist.
\end{enumerate}


\subsection{Formale Sprachen}
\label{sec:tg:fs}

\textbf{Defintion:} Wir nennen eine endliche nichleere Menge $\Sigma$ ein
\emph{Alphabet} und Elemente von $\Sigma$ Buchstaben. Der freie Monoid über
$\Sigma$ bezeichnen wir mit $\Sigma^*$ und das Neutralelement darin bezeichnen
wir mit $\lambda$


\subsection{Deterministische endliche Automaten}
\label{sec:tg:dfa}

% \textbf{Proposition:} Sei $M$ eine Halbgruppe und $x\in M$. Dann gibt es eine
% natürliche Zahl $n\geq 1$, sodass $x^n$ idempotent ist.

% \textbf{Korollar:} Ein Monoid ist genau dann eine Gruppe, wenn das einzige
% idempotente Element das Neutralelement ist.
\end{document}